\chapter{Methodology}
\label{ch:methodology}
The methodology applied in this master thesis project work is outlined in this chapter by showing the step wise progress desired for this project. The six steps are described in detail in this chapter.
%How do I want approach/solve the concrete goals...
%structure in sub chapters if possible

\section{Literature review}
As a very first step a preliminary study and literature survey to examine the state-of-the-art of existing work on the fundamentals WSN and IoT technologies, and the security protocols for lightweight key management and authentication. 
The research dealt with understanding the literature on proximity-based authentication techniques and group-key management, identifying then the research problems which is congruent to goal G1 presented in Section \ref{sec:goals}.

\section{Obtaining data from real-world use-case analysis}
After deciding the hardware to use for the data acquisition, the nodes have been implemented and programmed in order to get ambient data, which is in line with goal G2. 
The decision of the selected hardware was based on the availability of micro-controllers provided by the university and on the kind of data examined in the literature.


\section{Data analysis and authentication key generation}
Analysis of the ambient data in order to find patterns and attributes which could be exploited to generate a unique key between devices located in the same environment, corresponding to goal G3. 
A fuzzy commitment scheme has been then implemented based on the previous analysis. 


\section{Group-key establishment design and implementation}
Theoretical design of the lightweight group-key management scheme for resource-constrained sensor nodes in WSN and IoT and implementation on a simulated environment to fulfill goal G4.
Primitives proved to be secured or whose security relies on computationally hard problems are used as building blocks for the construction of the scheme.


\section{Conduct a performance evaluation}
In this phase was the evaluation of the proposed solutions using estimations and simulations in order to reach the goal G5.
The conducted performances evaluation is expressed in terms of the following metrics: 
\begin{itemize}
    \item Authentication performances
    \item Energy consumption 
    \item Computation overhead 
    \item Communication overhead
\end{itemize}




