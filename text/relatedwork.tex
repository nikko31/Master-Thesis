\iffalse
In recent, years biometric information is utilised to an in-creasing degree to replace or enhance classical cryptographic schemes \cite{Skoric2010SecurityData}.  Popular examples are photos or finger prints in ID-documents, Iris-scans or in the future probably short tandem repeats in a human’s DNA [29].  Generally, features are extracted from the biometric data and a characteristic set of features is required to match in o
\fi

The proposed protocol is based on a previous work presented in \cite{Gebremichael2018} and \cite{Ferrari2018}.
There are well studied group key establishment protocols in use today. 
An extension of the Diffie-Hellman protocol to a group of nodes with the generation of a one-time session key is described in \cite{Steiner1996}and \cite{Bresson2007}. However, the intensive computational power required for the execution of these protocols makes them not ideal for power and resources constrained devices.
Moreover, \cite{Bohli2007} demonstrated that \cite{Bresson2007} does not meet some security requirements.

A conference-key distribution system is proposed in \cite{Ingemarsson1982} but the the system resulted insecure because the information exchanged by the users makes it possible for a passive eavesdropper to compute the key. 
Moreover, the approach used for \cite{Ingemarsson1982}  requires a high number of messages exchanged and an high number of expensive computational operations.

Some of the earliest proposed protocols such as $\mu$TESLA \cite{Perrig2002} belongs to the symmetric key based protocols category and they are based on hash function in order to reduce the energy consumption. 
Other symmetric key schemes based on key ring like \cite{Bohli2007} are not scalable, and therefore, these are unsuitable for dynamic environments such in the analyzed scenario.

In \cite{Porambage2015} the authors propose two lightweight group-key establishment protocols using an approach similar to ours.
The first protocol allows only the legitimate members of the multicast group as eligible to continue the rest of the process of key derivation. 
This one is more appropriate for distributed IoT applications, which require nodes to contribute hightly to the key computation and need greater randomness. 
The second one allows to establish a shared secret key among the multicast group. This one is more suitable for centralized IoT applications, where a central entity performs the  majority of the cryptographic functions. 

For the purpose of authentication, most of the proposed solutions involve human interaction such as the Wi-Fi protected setup \cite{EldefrawyDynamicTimestamping} or the the necessity to use pre shared keys such as \cite{Gebremichael2019LightweightPad}. 
In recent, years biometric information is utilised to an in-creasing degree to replace or enhance classical cryptographic schemes \cite{Skoric2010SecurityData}.
Researchers started to explore context-based pairing protocol in order to capture commonly observed context features for pairing, leveraging on-board sensors and removing the "human-in-the-loop" factor.
A solution which uses ambient light or sound is proposed in \cite{Miettinen2014Context-BasedDescriptors}. 
In \cite{Schurmann2013SecureAudio} is proposed a solution which leverages audio for the secure pairing, but it results sensible to synchronization.
A solution based on heterogeneous context features is proposed in  \cite{Han2018DoTypes} and it relies on events observed by each sensor.